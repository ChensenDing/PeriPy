\documentclass{article}
\usepackage[utf8]{inputenc}
\usepackage[english]{babel}

\usepackage{amsfonts}
\usepackage{amsmath}
\usepackage{amsthm}
\usepackage{xcolor}

\newtheorem{thm}{Theorem}
\newtheorem{lem}[thm]{Lemma}

\usepackage[margin=0.7in]{geometry}



\title{Probabilisitic Peridynamics}
\author{Ben Boyes, Tim Dodwell, Mark Girolami}
\date{July 2019}

\begin{document}

\maketitle

\begin{abstract}
What does this paper do!
\end{abstract}
\noindent {\bf Keywords}: Fracture, Peridynamics, Stochastic Differential Equations, Bayesian Parameter Estimation

\begin{itemize}

\section{Introduction}

\item Peridynamics is a reformulation of classical mechanics, which gives rise to a nonlocal continuum model.

\item The key feature is formulation is written as integral representation of the force density around any material point. Consequently there are no assumptions of the differentiability of the displacement field. Hence, it's significant advantage over classical continuum models built around weak-derivatives of the stress tensor, peridynamic naturally handles the discontinuities which arise when modelling fracture mechanics.

\item Connection to O'Hagan, that the model mis-specification is not inputted onto the observation, but on to the model / process itself.

\item Output is we obtain a distribution of possible crack paths.

\item Framework by which to do parameter estimation for thorny hyper parameters defining peridynamics.

\end{itemize}



\section{Probabilistic Peridynamics}

In this section, we start by recapping the key concepts of the deterministic peridynamic formulation, and gradually introduce the reformulation into a probabilistic setting. Here the formulation presented is immediately the discretised version of the peridynamic model. The formulation as a integral partial differential equation, prior to adopting a solution strategy is well documented, see for example [Host of references].

\smallskip
Let a body in it's initial configuration occupy a domain $\mathcal B \subset \mathbb R^d$. In peridyanmics, computaionally the material can be represented by $N$ particles, each with the initial coordinates ${\bf x}_i \in \mathcal B$, displacement vector ${\bf u}_i(t) \in [0,T] \times \mathbb R^d$ and associated lumped volumes $dV_i$. Furthermore a given particle is subject to an externally applied force ${\bf p}_i \in \mathbb R^d$. At any point in time $t$ the position of a particle is
$$
{\bf y}_i(t) = {\bf x}_i + {\bf u}_i(t) \quad \mbox{for} \quad t > 0.
$$
The peridynamic formulation is a {\em non-local} model, since it accounts for interactions between material points at finite distance. Therefore, the $i^{th}$ particle interacts with the all particles within the index set
$$
\mathcal H_i := \{ j \; | \; \| \textbf{x}_{(i)} - \textbf{x}_{(j)} \|< \delta \}.
$$
The user defined parameter $\delta \geq 0$ is the {\em horizon}, which describes a charactieric length scale specific to the mechanics of the material.

\smallskip
In this contribution a bond-based peridynamic formulation is considered [REF]. The approach outlined is not limited to this constraint, and the re-formulation of state-based peridynamics approach into a probabilistic framework would follow the steps outline below. Here, we enjoy the relative simplicity of bond-based models, and focus our attention of exploring the novel probablistic elements of the contribution.  Thereofre, between a pair of particles $i$ and $j$, which reside in each others horizons, a non-linear connection define by the strain energy density $F_{ij}({\bf u}, \theta) > 0$ is made, where $\theta$ are model parameters. Each of which a described in the paragraphs which follow. The derivative $\nabla_{{\bf u}_i} F_{ij} = {\bf f}_{ij}$, gives the force vector ${\bf f}_{ij}$ the particle $i$ receives due to its interaction with particle $j$.

{\color{red} Definition of connecting function}

\noindent
The connection between two particles has an original length of
\begin{equation}
\ell_{ij} = \ell_{ji} = \| \textbf{x}^{(i)} - \textbf{x}^{(j)} \|_2
\end{equation}
at which the connection stores no strain energy ({\color{blue}Ask John this question since could contain pre-stress}). Under a general deformation of each particle the strain of the connection (stretch over original length) is given by
$$
\varepsilon_{ij} = \frac{\| \textbf{y}_j - \textbf{y}_i\|_2 - \ell_{ij}}{\ell_{ij}}.
$$
We define the strain / force density curve in the axial direction $\boldsymbol{\xi} = \textbf{y}_j - \textbf{y}_i$
\begin{equation}
f = \boldsymbol{\xi} \cdot \nabla_{{\bf u}_i}F_{ij}({\bf u},\theta) :=
\begin{cases}
c \varepsilon_{ij} \quad \mbox{for} \quad \varepsilon_{ij} < s_{00} - \beta \\
\mbox{cubic} \quad \mbox{for} \quad |\varepsilon_{ij} - s_{00}| < \beta \\
A \exp\Big(-\alpha (\varepsilon_{ij} - s_{00} - \beta)\Big) \quad \mbox{for} \quad \varepsilon > s_{00}
\end{cases}
\end{equation}
Whilst this seems a complex construction, the axial load density against axial strain response has a continuous derivative, therefore by design $F_{ij}$, is $C^2$. This is required in the analysis which follows in Lemma \ref{lem:wellDefined}. In practise we take $\alpha$ large and $\beta$ small, so we approximate the saw-tooth response and unloading path originally proposed by Silling [].

\smallskip
For a general configuration ${\bf u}$, the strain energy stored within the body $\mathcal B$, is given by
\begin{equation}\label{eqn:strainEnergy}
V({\bf u},\theta) = \sum_{i=1}^N \left[\left(\frac{1}{2}\sum_{j \in \mathcal H_i} {F_{ij}({\bf u},\theta)dV_j \right) - {\bf p}_i^T{\bf u}_i + \frac{1}{2}\epsilon\; {\bf u}_i^T{\bf u_{i} \right]
\end{equation}
The first term is the sum of (half the) strain energy in all connecting bonds with a given particle. The half, simply shares the energy equally between the two particles which make up the bond, preventing any double accounting. The second term is the work done by an external load $\boldsymbol{\lambda}_i$ acting on a particle. The final term, is an additional term with confines the problem, energetically penalising extremely large displacements in the physical case where a particle is completely de-bonded from all others. The parameter controlling the influence of this final term $\epsilon > 0$, can be chosen to be arbitrarily small. This additional term, over standard peridynamics will be required to show that the formulation of the probablistic peridynamics SDE has an invariant measure, Lemma ??, however it will have no physical implications to the simulated states shown in the numerical experiments, Sec ??. 



\smallskip
A model for the evolution of the elastic body, is to assume that the system evolves to a configuration which minimises the total potential energy $V({\bf u})$. It is therefore natural to define the deterministic gradient flow
$$
\eta \;\partial_t \textbf{u} = -\nabla V ({\bf u}) \quad \mbox{for} \quad t > 0 \quad \mbox{and} \quad \textbf{U}(0) \quad \mbox{given}.
$$
Where $\textbf{U} = [\textbf{u}_1,\ldots,\textbf{u}_n]^T$, is the complete solution vector, and $\eta$ is a parameter which has the physical interpretation as the damping/viscosity of the material. The value of $\eta$ is a measure of the responsiveness of the material to non-equilibrium in forces acting on it. In this contribution the effects of inertia are neglected, but could in general be included by defining the new variable.

\smallskip\noindent
The approach set out in this paper is to reformulate the deterministic equation as a Stochastic Differential Equation (SDE), we derived the equation for overdamped Langevin dynamics, given by

\smallskip
\begin{equation}\label{eqn:sde}
d\textbf{u} = -\eta^{-1}\mathcal{C}\nabla V (\textbf{u})dt  + \sqrt{2\mathcal C}\;d\textbf{B} \quad \mbox{for} \quad t > 0 \quad \mbox{and} \quad \textbf{u}(0) \quad \mbox{given}.
\end{equation}
\smallskip\noindent
where $\textbf{B}$ denotes $N$-dimensional Brownian motion in $\mathbb R^d$ and $\mathcal C$ defines a covariance function, which for this paper is taken as
\begin{equation}
\mathcal C({\bf x}, {\bf y}) = \exp \left( \frac{1}{\lambda}({\bf x}_i - {\bf x}_j)^T  \;\Sigma \;({\bf x}_i - {\bf x}_j)\right).
\end{equation}
It is well understood that the density of ${\bf u}(t)$, evolves in time according to the Fokker-Planck equation.  This is an exceptionally difficult equation to solve in higher dimensions, yet it's steady-state solution is available in closed form and is the density of the Boltzmann-Gibbs measure defined by
\begin{equation}\label{eqn:BoltzmannGibbs}
\rho_\infty({\bf u}) = \frac{1}{Z}\exp(-\gamma V({\bf u})) \quad \mbox{where} \quad Z:= \int_{\mathbb R^N} \exp(-\gamma V({\bf u}))\;d{\bf u}.
\end{equation}
 We now show that for the probabilistic peridynamics formulation presented the Boltzmann-Gibbs measure is well-defined, and the distribution of the solution ${\bf u}(t)$ converges exponential fast to it.
 
\begin{lem}\label{lem:wellDefined}
The Boltzmann-Gibbs measure \eqref{eqn:BoltzmannGibbs} is well-defined.
\end{lem}
\noindent 
\begin{proof}
Following [Roberts1996] it is sufficient to show that the potential energy $V({\bf u})$ is $C^2$, is confining in that $\lim_{|{\bf u}| \rightarrow +\infty} V({\bf u}) = +\infty$ and
$$
 \exp\Big(-\gamma V({\bf u})\Big) \in L^1(\mathbb R^M), \quad \forall \gamma > 0.
$$
The differentiability condition is implied implicitly through the differentiability constraint imposed on the strain energy densities for each bond given by $F_{ij}$ [EQREF]. The addition of the $\frac{1}{2}\epsilon {\bf u}_i^T{\bf u}$ term in \eqref{eqn:strainEnergy} also ensures that $V({{\bf u})$ is confining and bounded below. The later of which ensures $Z$ in \eqref{eqn:BoltzmannGibbs} is integrable. {\color{blue} Is this condition actually tighter than this? Does it have to be bounded below by zero, since it holds of any $\gamma$, so it for a given $\gamma$ can you bound it below?}
\end{proof}

\begin{lem}
The distribution of ${\bf u}(t)$, as defined by \eqref{eqn:sde}, converges exponentially fast to its Boltzmann-Gibbs measure \eqref{eqn:BoltzmannGibbs}.

\end{lem}
\begin{proof} Alongside Lemma \ref{lem:wellDefined}, it is sufficient to show that $V({\bf u})$ satisfies the following inequality for some $0 < d < 1$
$$
\liminf_{|{\bf u}| \rightarrow +\infty} \Big[{(1 - d)|\nabla V({\bf u})|^2 - \gamma^{-1}V({\bf u})}\Big] > 0.
$$

{\color{red} TO DO!}
\end{proof}










\end{document}
